\begin{abstract}
    Negli ultimi dieci anni, il panorama dell'informatica è stato notevolmente influenzato dalle discipline conosciute come \textit{Big Data} e \textit{Machine Learning}.
    L'ampia digitalizzazione dei servizi, la diffusa adozione di reti sociali, gli acquisti online, l'uso diffuso di dispositivi indossabili avanzati, e la crescente automazione nei settori industriali, come la produzione e la logistica, rappresentano solo alcune delle forze trainanti di un'enorme crescita dei dati a nostra disposizione.
    Queste enormi quantità di dati hanno messo in evidenza i limiti delle tradizionali metodologie di gestione dei dati, spingendo verso l'adozione di nuovi approcci più efficienti, che rientrano nell'ambito dei \textit{Big Data}. \\
    Questi nuovi approcci non solo sono in grado di gestire grandi quantità di informazioni, ma forniscono anche strumenti avanzati per scoprire modelli e relazioni nascoste, aprendo nuove prospettive di esplorazione dei dati.
    Inoltre, tali approcci si basano su architetture altamente scalabili che possono adattarsi alle esigenze specifiche e al volume di lavoro trattato. \\
    Parallelamente, il settore dell'intelligenza artificiale noto come \textit{Machine Learning} ha visto una notevole espansione al di fuori dei tradizionali contesti di ricerca.
    L'evoluzione dei veicoli autonomi e dei robot, la medicina personalizzata e la diagnosi predittiva, oltre all'automatizzazione avanzata dei processi decisionali aziendali, sono solo alcune delle applicazioni in cui il \textit{Machine Learning} ha dimostrato il suo valore. \\
    L'introduzione del \textit{Cloud Computing} ha ulteriormente rivoluzionato questo settore, consentendo l'utilizzo di risorse di calcolo virtualmente illimitate e delegando la complessità della gestione hardware e software ai fornitori di servizi cloud. \\
    Questo progresso e l'interesse crescente per queste discipline hanno dato origine a una nuova figura professionale denominata \textit{Data Scientist}, che rappresenta una combinazione di competenze in informatica, statistica, matematica e domini specifici dell'industria.
    Questi esperti sono essenziali per sfruttare appieno il potenziale dei dati e tradurli in informazioni significative per la strategia aziendale.
\end{abstract}
